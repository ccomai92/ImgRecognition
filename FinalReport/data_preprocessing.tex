The original data sets are organized in the .csv file.
The objects are consist of attributes such as \textit{font} (string), \textit{fontVariant} (string), \textit{label}(int),
\textit{strength} (int), \textit{italic} (int), \textit{orientation} (int), \textit{m top} (int), \textit{m left} (int),
\textit{originalH} (int), \textit{originalW} (int), \textit{h} (int), \textit{w} (int), and 20 by 20 size grey scale image matrix.
Most of the attribute names are self-explanatory. However, to be more specific,
\textit{m top} and \textit{m left} are coordiate where image of the character starts in the real image in terms of row and column.
Moreover, \textit{originalH} and \textit{originalW} are original image size. Finally, \textit{h} and texit{w} are the height and weidth of
the image.
\newline
\newline
\indent
To begin with, some of the attributes are omitted in this process because recognizing font and whether the
character is italic are not the part of this project. Moreover, all the character images are original image.
In other words, image in the datasets only contains one character. Therefore, \textit{label} of the character and the
20 by 20 size grey scale image is going to be used in this project.
\newline
\newline
\indent
Next step is to convert the grey scale matrix into the RGB matrix which is three dimensional. The result matrix
after preprocessing would produce 20 by 20 by 3 image matrix. To be specific, conversion took place by appending
the original grey scale matrix into the result matrix three times. The result is the N number of RGB image matrices.
\newline
\newline
\indent
Furthermore, data preprocessing step produces a vector that values in each index contains character label and this
corresponds the image matrices stored in the image matrices. Finally, this step produces a matrix size of N by the number
of unique characters. The value in the indices are initially zeros in floating point. After image conversions, based on
the image label, vector for each image will contain 1.0 where index corresponds the image label. For instance, if there are
five different unique image labels in 10 different objects, the result will be ten by five matrix containing 1.0 in the vector
where index matches the character label, otherwise values will be 0.0s. In this case, the floating point one value specifies that
the index, character label, is true for corresponding image.